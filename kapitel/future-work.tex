\chapter{Future Work}
\label{future-work}
In this work we presented a novel approach which improves the chances of reusing software by seizing continuous integration process. The approach seems promising as developers do not need to spend much effort on searching reusable component as the prototype, Simile, does this automatically in the background. However, there still exist many aspects with potential for improvement.

Currently our proof-of-concept supports only projects developed in Java, so it would be interesting to support other programming languages such as Javascript, Scala, or Go. 

Support other CI servers and VCSs is a task to be done in the future. In the ongoing version of Simile, it supports only Jenkins as CI server therefore the support of other CI software however to support other alike tools is not a difficult task. This is due to the fact that it is only necessary to implement plugins with the functionality of send information to Simile. Nevertheless to support other VCSs is more complex as it will require changes in the core of Simile.

Although the time taken by code-search engines and rankings to deliver a result have been reduced, it is still considerable to make test-driven reuse applicable for the daily work of a developer.

Further testing must be done in order to evaluate the usefulness of the approach. For our proof-of-concept we used a dummy project which implements Base64 as described in Chapter \ref{usage-scenario}. Therefore it would be interesting to apply Simile in a real project to analyze information. For instance data like number of components retrieved, number of components reused, or time taken to retrieve a result would be interesting to measure.

Enhance the features provided by this prototype is another task to be considered. Currently the request to SOCORA for searching component using test-driven approach supports three non-functional requirements: leanness, operations/second, cohesion, and efferent coupling. These are by default and they cannot be change by the user. Therefore, for a future version more non-functional requirements should be supported such as superfluous functionality, needed functionality, balance, functional sufficiency, among others. Send the result to more emails at the same time is another feature to be implemented from now on.

Despite the fact that SOCORA, which our proof-of-concept relies on for searching components, is in a work-in-process status, the team behind is working on enhancing features provided. Integration into desktop IDEs, inclusion of of further metrics and quality models specially tailored to pragmatic reuse, the accommodation of more relaxed filtering criteria in the result set and providing guidance on the nature of the test that should be supplied as search queries, are some of the features planed to be implemented in SOCORA\cite{Kessel2016}.