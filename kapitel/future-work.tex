\chapter{Future Work}
\label{future-work}
The approach presented in this thesis seems promising as developers do not need to spend very much effort on searching reusable component as Simile does this automatically in the background. However, there still exist many aspects with potential for improvement.

Currently our proof-of-concept supports only projects written in Java programming language, so it would be interesting to support other programming languages such as Javascript, Scala, or Go. 

Supporting other CI servers and VCSs is a task to be done in the future. In the ongoing version of Simile, it only supports Jenkins. The inclusion of other alike tools is not a difficult task because it is only necessary to implement plugins for each new tool with the functionality of sending the relevant information (i.e. repository URL, branch, and email) to Simile. Nevertheless it is more complex to support other VCSs as it would require changes in the core of Simile.

Although the time it takes to deliver a result by the code-search engines and rankings has been reduced, it is still considerable to make test-driven reuse applicable for the daily work of a developer. Therefore further enhancements need to be done in this regard.

Further testing must be done in order to evaluate the usefulness of the approach. For our proof-of-concept we used a dummy project which implements Base64 as described in Chapter \ref{usage-scenario}. Therefore it would be intriguing to apply Simile in a real project to analyze information. Data like numbers of components retrieved, number of reused components, or time taken to retrieve a result would be interesting information to measure.

Enhancing the features provided by this prototype would be another task to be considered. Currently the request to SOCORA for searching component using test-driven approach supports three non-functional requirements: leanness, operations/second, cohesion, and efferent coupling. These are by default and cannot be modified by the user. Therefore, for a future version more non-functional requirements should be supported such as superfluous functionality, needed functionality, balance, functional sufficiency, among others. Sending the result to more emails at the same time is another feature to be implemented from now on.

Despite the fact that SOCORA, which our proof-of-concept relies on for searching components, is in a work-in-process status, the team behind is working on enhancing provided features. Integration into desktop IDEs, inclusion of further metrics and quality models specially tailored to pragmatic reuse, the accommodation of more relaxed filtering criteria in the result set and providing guidance on the nature of the test that should be supplied as search queries, are some of the features planed to be implemented in SOCORA\cite{Kessel2016}.