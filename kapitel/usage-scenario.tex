\chapter{Usage Scenario}
\label{usage-scenario}
Before explaining the approach we will propose in the next chapter, we will explain a scenario that illustrates the opportunity found that motivated our approach to improve software reuse leveraging continuous integration. The following scenario is based on the usage scenario found in \cite{Kessel2016}.

Envisage the following scenario in which a development team is working or has code ownership of a web service application, and recognizes that she/he requires the ability to encode and decode information in the common Base64 format (e.g., the Base64 variant used in MIME transfer encoding \cite{rfc2045}). Also this team has adopted continuous integration, thus every member of the team is committing code at least once a day and the code committed is tested before it is accepted by the VCS. We will assume that the team is using Git as VCS, Jenkins as CI server, Java as main language and that the team has adopted TDD.

From this simple scenario emerge the following questions:

\begin{description}[style=nextline]
\item[How might the team realize that what they are working on has not been developed already?] \hfill \
If we follow the example, the implementation of encode and decode in Base64 is very easy to find, however that does not mean that the team knows that at the moment they are developing. Therefore it would be very useful that an automatic tool analyzes the code and extracts the method signature so it would be able to make a code search for similar methods.
\item[How might we seize continuous integration in order to improve software reuse?] \hfill \
Continuous integration seems to be a very good opportunity for software reuse. According to CI workflow explained in \ref{sec:ci-def} every change made in the code should be committed and pushed to the VCS. This pushed code is tested by the CI server and accepted if it passes all the tests otherwise it is rejected. This is a good chance to extract method signatures and make the code search named previously.
\item[Would TDD be helpful?] \hfill \
According to TDD process before implementing a method a failure test should be implemented before. If we assume that and that every code change is committed and pushed to the VCS, we could also extract the test classes to search for similar components.
\end{description}