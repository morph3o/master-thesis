\chapter{Conclusion}
\label{conclusion}
In this work we have presented a novel approach which is an attempt to improve the chances of reusing software components by seizing the continuous integration process. As a proof-of-concept we implemented Simile, which is a service that analyzes code, extracts interface signatures and test classes, and send them to SOCORA for searching similar components. Once SOCORA responds with the components found, this tool sends an email with details of the candidates to the recipient.

Several benefits come with this approach. Development cost and time development can be reduced by reusing software components. However some human-related challenges might jeopardize these benefits. For instance, developers need to be aware of the existence of components that might fit to its requirements. For this reason, the approach presented in this work is another way to improve the successful implementation of software reuse practices in a project. This is due to the fact that Simile is able to work in parallel to the software development process, thus it is able to recommend components with almost no effort from the developers to find and evaluate candidate reusable components. Moreover, by using test-driven search and SOCORA ranking, this tool can recommend components that fully meet the functional requirements ranked based on non-functional requirements.

Despite the fact that this approach is a contribution, it is still immature and needs further improvements. Firstly, our proof-of-concept supports only Jenkins, whereas other popular CI servers should be supported in the future. Secondly, supporting the selection of other non-functional requirements by the user needs to be implemented. Thirdly, test-driven search approach necessitates further refinements to improve the time taken to return a result. Finally, SOCORA is a on-going project which will be improved in future versions by supporting more features which ultimately will improve the outcomes.


