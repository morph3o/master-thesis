\section{Microservices}
Microservice is a software architecture which is usually linked to Martin Fowler in its article \cite{Fowler2014}. It is an architectural style that aims to develop applications as a suite of small services independently from each other. Each of these services runs in its process and communicate with lightweight mechanisms (e.g. HTTP resource API). These services are built around business capabilities and are independently deployed by fully automated deployment machinery. There is furthermore a bare minimum of centralized management of these services, which may be written in different programming languages and which use different data storage technologies \cite{Fowler2014}.

\begin{figure}[htb]
	\centering
    \includegraphics[width=0.2\textwidth]{grafiken/monolith-app}
    \caption{Monolithic Architectural Style}
    \label{fig:monolith}
\end{figure}

This style emerges as a solution to the problems of monolithic style. An application built using monolithic architectural style is considered as a single unit with normally three main parts (Figure \ref{fig:monolith}): a client-side user interface (i.e. HTML, CSS, JavaScript running on user's machine), a database and a server-side application. The latter is a monolith as it handles HTTP request, execute domain logic, retrieve and update data from the database, and populates and/or select the HTML views sent to the client-side. As it is a single logical executable, any changes made in the system, no matter the size of it, require a build and a deploy of the whole server-side application. It is hard to keep a good modular structure, making it harder to keep changes that must affect one module within that module. Also it is hard to scale as it is necessary to scale the entire application rather than parts of it requiring greater resources. These problems are causing frustrations on people that are using this style, specially nowadays when applications are deployed to the cloud, leading to the microservice architectural style.

Microservice architectural style, depicted in Figure \ref{fig:microservices}, proposes to split the software into independent services which makes it easy for changes as they are independently deployable components. Furthermore services have more explicit component interfaces by using explicit remote call mechanisms. Moreover, this approach prefers letting each service manage its own database instead of using a central database for all the services. Applications developed with this style have meany other benefits such as single responsibility, high scalability, easy to enhance, and ease of deployment. In the following subsection we will discuss about benefits, challenges and common characteristics that can be found in applications developed following this architectural style.

\begin{figure}[H]
	\centering
    \includegraphics[width=0.6\textwidth]{grafiken/microservices-app}
    \caption{Microservice Architectural Style}
    \label{fig:microservices}
\end{figure}

\subsection*{Characteristics of Microservice Architecture}
The work of Fowler and Lewis \cite{Fowler2014} describes a list of common characteristic of applications built using microservice architectural style. It is important to point out that all applications using this style must confirm to every characteristic of this list.

\begin{description}[style=nextline]
\item[Componentization via Services] \hfill \
First of all it is important to explain the difference between libraries and services as the primary way of componentizing is by breaking down software into services. On the one hand libraries are components which are linked into an application and called using in-memory function calls. On the other hand services are out-of-process components which communicates with a mechanism such as a web service request, or a remote procedure call \cite{Fowler2014}.

Splitting a software into services brings benefits because services as components are independently deployable making them easy to change, likewise services have more explicit component interfaces when using explicit remote call mechanisms. 
\item[Organized around Business Capabilities] \hfill \
Normally in monolithic applications the development teams are divided by technology layers such as front-end team, server-side team and database team. However when team are divided in this way, even simple changes can lead to a cross-team project requiring time and budgetary approval.

In the microservice approach this is taken a different way by dividing the application into services organized around business capabilities. Thus services combine everything needed for that business area, including user-interface (UI), database, business logic, and any other external collaboration. Therefore each team is cross-functional, including the full range of skills required for the development of it.
\item[Products not Projects] \hfill \
Usually software development follows a project model where the objective is to deliver some piece of software which, when it is done, is delivered to a maintenance team and the development team is disbanded.

The microservice approach follows the product model instead of a project model. In the product model the development team takes full responsibility of the software, in other words it owns the product. This results in day-to-day contact between the developers and the users as they will have to take on at least some of the support. Moreover this mentality ties in with the linkage to business capabilities as the focus is on how the software can assist its users in enhancing the capabilities instead of putting the focus on a list of functionality.
\item[Smart endpoints and dumb pipes] \hfill \
Microservice uses the approach of smart endpoints and dumb pipes. Instead of putting so much smart into the communication mechanism itself, microservice style aims to be as decoupled and as cohesive as possible. In other word, a microservice works as a black-box which receives a request, applies a logic, and produces a response. Usually two protocols are used, HTTP request-reponse with resource API's and lightweight messaging (e.g. RabbitMQ, ZeroMQ).
\item[Decentralized Governance] \hfill \
Using a centralized approach brings consequences such as standardization of single technology which keeps the team of using the right tool for the job. In contrast to that the microservice approach allows the use of technology which fits the requirements best. For instance, if NodeJS is more appropriate for a service than Java then there will be no problem choosing it.
\item[Decentralized Data Management] \hfill \
Microservice approach prefer letting each service manages its own database, of either different instances of the same database technology, or of entirely different database systems - an approach which is called \emph{Polyglot Persistence} \cite{Leberknight2008}. 

Even though this approach has some implications for managing updates or increment of complexity, the benefits make it worth it. The common approach to solve the former implication would be to use transactions but implementing it into services is difficult, which is why microservice architectures emphasize transactionless coordination between services, assuming that consistency may be only eventual and problems are dealt with by compensating operations. This is useful in order to respond quickly to demands. In spite of the fact that using different technologies increases the complexity, it allows the teams to solve the problem with the right tool. For instance, if a service only needs to retrieve pages by looking up its ID, a key-value database would be more suitable than a relational database. Moreover, as many NoSQL databases are designed to operate over clusters, they can tackle larger volumes of traffic and data which is harder with vertical scaling \cite{Fowler2011}.
\item[Design for failure] \hfill \
A consequence of using microservice architecture, is that applications need to be designed so that they can tolerate the failure of services. As any service call could fail due to unavailability of the supplier service, the client's service must respond as neat as possible.

Although this can be considered as a disadvantage in comparison to monolithic approach, using good monitoring and logging mechanism would help to overcome this. For instance a dashboard which display the status of the different services, current throughput, and latency would help to react in case one or more services went offline.
\end{description}

In conclusion, although microservice architectural style comes with many benefits, it also entails several difficulties. By splitting an application up into services we can simplify and speed up the release process, we can use the right tool for a job, or focus on how to assist the user better instead of concentrating on a set of functionality which might not be of any value to the user. However we have to be aware that changes to one service might break its consumer, or that it might lead to an increment of complexity.

It is important to keep in mind that this style is not a \emph{silver bullet}. Fowler described in his blog post \cite{Fowler2014} that the core of The Guardian website \footnote{The Guardian: \url{https://www.theguardian.com/} (accessed: 04.02.2017).} was built as a monolith, however every new feature has been added as a microservice which uses the core's API.